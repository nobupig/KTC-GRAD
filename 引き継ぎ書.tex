📘 Firebase 成績入力システム
完全引き継ぎ書(2025-12 現在)
1️⃣ プロジェクト概要

対象:近大高専 成績入力システム

技術構成

フロント:HTML / CSS / Vanilla JS

バックエンド:Firebase Firestore

利用者:教員(担任・担当教員)

主目的:

教員が成績を入力

平均点・調整点を自動算出

最終成績を自動計算

後続で教務提出用データを生成

2️⃣ 現状の実装到達点(STEP B〜C-2 完了)
✅ 基本構造

score_input.html

科目選択

組・コース(全員 / 1〜5)切り替え

成績一覧表示

JS 分割構成

score_input_loader.js:全体統制・イベント管理

score_input_students.js:名簿生成・フィルタ

score_input_modes.js:素点 / 自動換算 / 最終成績計算

score_input_paste.js:Excel貼り付け

score_input_criteria.js:評価基準ロード

fetch_isSkillLevel.js:科目属性取得

✅ 成績入力機能

入力方式

手入力

Excel貼り付け

入力制御

数値のみ

即時最終成績反映

自動保存は 未実装(今後)

✅ 平均点・調整点(STEP C-2.1 完了)
🔹 平均点

常に「全体(全受講者)」を母集団として計算

組・コースでソートしても 平均点は変化しない

分母は:

「現在入力されている値のみ」

空欄は除外

入力のたびに リアルタイム更新

🔹 調整点

利用条件(※修正済・確定)
以下の いずれかを満たす科目:

subjects.passRule === "adjustment"

subjects.required === true

計算式(確定):

調整点 = ceil( 平均点 × 0.7 )


平均点が更新されるたびに リアルタイム連動更新

組・コース切替では変化しない(全体固定)

✅ 科目属性の参照先(重要)

正:subjects コレクション

誤:teacherSubjects_YYYY

⚠️ teacherSubjects_YYYY には
passRule / required / isSkillLevel が 存在しない

→ 調整点判定・赤点判定は 必ず subjects を参照すること

3️⃣ 確定している仕様(今後も変更しない前提)
🔒 表示仕様

平均点・調整点は

常に全体値

ソート・フィルタの影響を受けない

組・コース切替は 表示用のみ

🔒 データ仕様

Firestore ルール(既存)

教員のみ read/write

科目・学生は read only

成績データは年度別コレクションで管理予定

🔒 UI仕様

平均点・調整点は

科目選択の右横に表示

成績一覧は

Excelライク

スクロール可能

ヘッダ固定

4️⃣ 未実装・次にやるべき STEP
▶️ STEP C-3(次の作業)
🔹 赤点・超過ロジック

判定ルール(予定)

調整点使用科目

最終成績 < 調整点 → 赤点


非使用科目

最終成績 < 60 → 赤点


表示

赤点:セル背景色変更

超過学生との併存表示対応

▶️ STEP D

Firestore 保存処理

成績

最終成績

調整点

赤点フラグ

超過フラグ

保存単位・トランザクション設計

▶️ STEP E

教務提出用データ生成

Spreadsheet / Excel

指定フォーマット

5️⃣ 注意点(今回ハマったポイント)
⚠️ ① 平均点が壊れる原因

表示中の行だけで再計算してしまう

対策:

計算対象は常に

全 studentsData(全体)


DOMの表示行を使わない

⚠️ ② ソートイベントで再計算してしまう

組・コース切替時に

recalculateAverage() を呼ばない

平均点は

入力イベント時のみ更新

⚠️ ③ 科目属性の参照ミス

teacherSubjects_YYYY を見ていたため

調整点が表示されなかった

必ず subjects を参照

⚠️ ④ Copilot誤爆対策

1 STEP = 1 目的

「変更してはいけない仕様」を必ず明記

関数名重複(例:updateAveragePointDisplay)に注意

6️⃣ 次チャットでの開始文(推奨)
上記の完全引き継ぎ書の続きとして開発を再開してください。
次は STEP C-3(赤点・超過判定と表示)から進めたいです。