📘 **KTC 成績処理システム(Firebase版)

完全引き継ぎ書(2025/12/08 最新版)**

🔵 1. システム全体構成(Firebase 完全移行)

本システムは GAS 版→Firebase 版へ全面移行した新しい成績処理プラットフォーム。

採用技術:

Firebase Authentication(Google ログイン)

Firestore(NoSQL)

Firebase Hosting

HTML + Vanilla JavaScript (ES Modules)

Firestore Security Rules(高度セキュリティ)

認証:

@ktc.ac.jp の教員のみログイン可能

students / external は正規表現で拒否

teachers コレクション登録済メールのみ通過

🔵 2. Firestore データ構造(確定版)
✔ teachers / {email}

教員名簿(ログイン許可の基準)

✔ teacherSubjects_2025 / {email}

その教員が担当する科目の一覧
(担当解除=ここから削除)

✔ subjects / {subjectId}

科目マスタ(学年・コース・教科名などをまとめた ID)

✔ evaluationCriteria_2025 / {subjectId}

📌 今年度の「科目の評価基準」(subjectId 一意)

もはや「教員×科目」のデータではない

科目に対して1つだけ存在

教員が変わっても評価基準は科目に属する

修正理由必須

合計値 100 ±1% チェック

評価方式(考査のみ / 考査+課題 / 実験)と項目行の管理

✔ evaluationLocks_2025 / {subjectId}

評価基準の編集排他制御

編集開始時 → lock

科目切替/ページ離脱時 → unlock

他教員が編集中の場合 → 編集不可(閲覧のみ)

✔ evaluationHistory / {autoId}

評価基準の修正履歴

before / after

編集者 email

修正理由

時刻
※ オーナー・管理用ログとして残す

✔ drafts / {id}

成績入力画面側で利用予定

✔ settings / evaluationPeriod

評価基準入力期間・成績入力期間を後で実装可能

🔵 3. subjects.html(担当科目管理)の仕様(最新版)
✔ 科目登録

学年・コース選択→科目読込→担当科目として登録

Firestore: teacherSubjects_2025/{email}

✔ 科目削除(超重要)

確定仕様:担当者が 0 人になったら「評価基準を削除」する

状況	削除される?
A先生だけが誤登録 → 評価基準入力 → 削除	✔ subjectId の evaluationCriteria を削除
A先生が削除したが、B先生が同科目を担当している	❌ 削除しない(科目として有効)
全教員が担当解除した	✔ 評価基準を削除(正しい)

これにより:

誤登録→誤評価基準入力→削除
→ 評価基準が自動クリア

正しい教員が後から登録しても
→ 評価基準が真っさら(混乱ゼロ)

✔ 履歴(evaluationHistory)は削除しない

科目の修正履歴として残すのが正しい

基準削除と履歴削除は別の責務

✔ UI(削除時の警告)は変更済み

「教務へ連絡してください」は削除
「担当解除されます」に変更可能

🔵 4. evaluation.html(評価基準入力画面)
✔ 新仕様

評価基準は subjectId 一意で管理

✔ 主機能

科目選択

評価方式選択(UI タブ形式推奨)

項目行追加/削除

100% チェック

保存(Firestore 書き込み)

保存後はロックしたまま UI を編集不可

「修正する」押下でロック更新→再編集

修正理由必須

修正履歴を保存

ロック解放はページ離脱/科目切替など

✔ UI

ログイン表示

評価方式タブ(折り返し無し)

スタイリッシュなバナー(複数教員担当の注意表示)

保存、修正、警告色など改善済み

🔵 5. 評価基準削除ロジック(今回確定した新仕様)

deleteSubject(subjectId) の流れ:

教員 A が担当解除 → teacherSubjects_2025/{A} から subjectId を削除

teacherSubjects_2025 全教員を走査して

subjectId を持つ教員数をカウント

count === 0 の場合のみ
→ evaluationCriteria_2025/{subjectId} を削除

履歴(evaluationHistory)は残す

他教員の担当がある場合は評価基準は残す(正しい動作)

🔵 6. Firestore Security Rules(現在の要点)

teachers に存在する email のみ read/write

subjectId 一意の構造に対応済み

evaluationCriteria_2025 と evaluationLocks_2025 は write 許可

students / 無関係ユーザーは全部拒否

🔵 7. テストアカウント

学内メール以外でも temporarily allow 設定可能(テスト期間中のみ)
→ 本番移行時に除去

🔵 8. 現状までで実装済み・完全動作が確認できたもの

ログイン(Auth)

教員名簿チェック

科目登録 UI

科目削除(担当解除+担当0人時の基準自動削除)

評価基準入力画面(subjectId 一意)

ロック制御

修正履歴

UI の改善

全ページ連動のログイン中表示

評価方式の機能・UI

100% チェック

教員が誤登録した評価基準の自動クリア(今回の新仕様)

🔵 9. 次に進むべきステップ(あなたのプロジェクトの現状から最適ルート)

ここからの開発は以下の順番が最も安全で効率的です:

⭐ STEP 1:成績入力画面の設計(最重要)

いよいよ本丸。

evaluationCriteria_2025 に基づき入力フォーム自動生成

行の項目名+割合に従って成績欄生成

drafts(途中保存)

保存済ステータス

ロック

最終送信

科目切替と UI リセット

正しい書式(成績一覧のサマリも含む)

次チャットでこの設計を開始するのが最適。

⭐ STEP 2:成績期間(期末評価期間)ロック

settings の evaluationPeriod に基づき

期間外は入力不可

修正不可

閲覧のみ

⭐ STEP 3:教務用管理画面(未入力科目一覧)

“評価基準未入力の科目一覧”

“成績未入力の科目一覧”

授業担当全体の進捗確認

運用上、必ず必要になる。

⭐ STEP 4:成績の PDF エクスポート

教務が教員に「評価基準 PDF 出して」と言われるケースが多い

ロジック自体は簡単(Firestore → HTML → jsPDF)

⭐ STEP 5:ロック自動解除(Cloud Functions)

24時間以上残ったロックを削除

事故防止

無料枠で運用可能

🔵 10. 引き継ぎ書の使い方

次チャットにこのまま貼るだけで:

現状を 100% 継承できる

仕様の誤解が防げる

次ステップの設計に即入れる

作業漏れを防止できる

あなたのプロジェクト全体を完全に再構築できる内容になっています。