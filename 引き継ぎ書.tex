📘 **KTC 成績システム(Firebase版)

科目一覧インポート工程:完全引き継ぎ書(2025/12/04 最新版)**

🟦 1. 全体概要

本工程では、KTCの「科目一覧スプレッドシート」を、
Google Apps Script(GAS)を用いて Firestore に subjects コレクションとして取り込む仕組み を構築し、
392 科目のうち数件をテストインポート → 本番全件インポートまで正常に完了した。

本作業は、今後の「科目登録画面」「成績評価基準入力」「成績入力画面」を構築するための 基盤データ整備 の役割を持つ。

🟦 2. Firestore 構造(subjects コレクション)

Firestore に登録される1科目分のデータ構造は以下:

subjects/{docId} {
  name: "測量学",
  grade: "3",
  course: "CC",                 // ★ C/CC/CA を区別して保存(重要)
  semester: "前期",

  required: true,                // 必修=真
  useAdjustment: false,          // 調整点使用有無(〇 → false)
  passRule: "fixed60",           // fixed60 / adjustment
  fixedPassLine: 60,             // 合格基準(調整点使用しない科目)

  specialType: 0,                // 特別科目 1/2 の区別
  isSkillLevel: false,           // 習熟度科目判定

  teacherList: [],               // まだ空(後の科目登録で使用)

  createdAt: "timestamp"
}

🔥 特記事項(超重要)
✔ コース(course)は C/CC/CA を統合しない

C → 通常都市環境

CC → 都市環境(クラスB)

CA → 都市環境(クラスA)

成績入力で学生名簿の取り扱いが異なるため、
元の値のまま Firestore に保存する必要がある。

🟦 3. 科目一覧スプレッドシート(列構造)
列番号	列名	説明
0	科目名	例:数学Ⅰ
1	開設学年	1〜5年
2	組・コース	G, M, E, I, C, CC, CA
3	開講期	前期 / 後期 / 通年
4	履修区分	必修 or 選択
5	調整点	〇 → 調整点なし(合格60固定)
6	特別科目	1/2(成績処理方法が異なる)
7	習熟度	〇(特殊扱い)

この構造を完全に解析し、Firestore 用に正規化して保存。

🟦 4. Firestore セキュリティルール(現時点の運用版)

以下ルールを設定し「公開(デプロイ)」済み。

rules_version = '2';
service cloud.firestore {
  match /databases/{database}/documents {

    // 1. teachers:ログイン済みユーザーは読み取り専用
    match /teachers/{email} {
      allow read: if request.auth != null;
      allow write: if false;
    }

    // 2. subjects:GAS(認証済み OAuth)とログインユーザーのみ read/write
    match /subjects/{docId} {
      allow read, write: if request.auth != null;
    }

    // 3. その他は拒否
    match /{document=**} {
      allow read, write: if false;
    }
  }
}

✔ セキュリティ上のポイント

teachers は書き換え不可(攻撃されても安全)

subjects はログイン(Auth)済ユーザーのみ可

外部からの匿名アクセスは全て拒否

🟦 5. GAS 側の設定(appsscript.json)

Firestore 書き込みに必要な OAuth スコープを追加:

{
  "timeZone": "Asia/Tokyo",
  "dependencies": {},
  "exceptionLogging": "STACKDRIVER",
  "oauthScopes": [
    "https://www.googleapis.com/auth/script.external_request",
    "https://www.googleapis.com/auth/spreadsheets",
    "https://www.googleapis.com/auth/userinfo.email",
    "https://www.googleapis.com/auth/datastore"   // ★FireStore書き込み用(必須)
  ]
}

✔ “auth/datastore” がないと Firestore 書き込み 403 になる

今回の失敗→成功はここがポイント。

🟦 6. GAS と Firebase プロジェクトの紐付け

Apps Script のプロジェクトは
Firebaseプロジェクト(ktc-grade-system)へ接続(Google Cloud プロジェクト変更) した。

これにより:

Cloud Firestore API が有効

OAuth トークンが「Firebaseプロジェクトとして有効」

になり、書き込み権限が整った。

🟦 7. インポートスクリプトの完成版(重要部分のみ)

以下の GAS コードにより科目を Firestore に投入。

主なポイント:

C/CC/CA を normalize しない

特別科目・習熟度・調整点を正確に解釈

Firestore ドキュメントIDは
grade_course_semester_name
形式で一意に生成

teacherList は空配列(後で科目登録画面から登録)

※コード全文は前チャット参照(有効確認済)

🟦 8. テストインポート → 本番インポートの流れ

appsscript.json を修正

GAS を再認証

インポート対象を 5 行だけに制限

Firestore subjects に正しく追加されたか確認

全件インポートへ切替

392 科目中、全件正常に投入

🔥 結果:全て Firestore に正しく取り込み成功

🟩 9. この段階での成果(非常に重要)

Firestore に KTC版科目マスタが構築完了

教員側の科目登録画面で利用するデータが揃った

成績評価基準入力・成績入力に必要なデータ構造が確定した

これにより、
次の工程「科目登録画面 subjects.html」へ進む準備が完全に整った。

🟦 10. 次のステップ

次チャットでは以下のどれから開始するか選べます:

A. 科目登録画面(subjects.html)をスタイリッシュに実装

学年→コース制御

都市環境 C/CC/CA 統合表示

前期/後期/通年のフィルタ

表から複数科目に✔選択登録

確認モーダル

科目が見つからない場合 → 教務へメール

Firestore teacherSubjects への保存

B. UI ラフデザイン → 仕様書先に作成
C. Firestore セキュリティルールの先行構築
🎉 以上が「完全引き継ぎ書(最新版)」です。

必要なら PDF 版も作成できます。
次チャットでは 科目登録画面の実装 に進みましょう!