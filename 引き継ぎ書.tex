。

📘 KTC 成績処理システム(Firebase版)完全引き継ぎ書(2025/12/07 時点)
1. システム全体構成(Firebase版)

本システムは Google Apps Script(旧システム)から完全 Firebase 化へ移行した新成績処理システム。

🔧 利用技術

Firebase Authentication(Googleログイン)

Firestore(NoSQL)

Firebase Hosting

HTML + Vanilla JS(Modules)

Firestore Security Rules(高セキュリティ実装済)

2. 認証仕様(完成済)
✔ 教員ログインのみ許可

@ktc.ac.jp 以外はすべて拒否

学生メール(g〇〇@ktc.ac.jp / s〇〇@ktc.ac.jp)は正規表現で拒否

Firestore の teachers コレクションに存在するメールアドレスのみログイン可能

✔ ログイン後の遷移
login.html → start.html

3. Firestore データ構造(確定)
teachers / {email}                 ← 教員名簿
subjects / {subjectId}             ← 科目マスタ
teacherSubjects_2025 / {email}     ← 教員の担当科目
evaluationCriteria_2025 / {id}     ← 評価基準(本年度)
evaluationHistory / {id}           ← 過去ログ
drafts / {id}                      ← 途中保存
settings / {id}                    ← 成績期間設定
notifications / {id}               ← 教務への通知ログ

4. Firestore セキュリティルール(最新版)
✔ 教員ログインしていて

かつ

✔ teachers コレクションに存在する教員

だけが read/write 可能。

すべてのコレクションは年度名・用途ごとに厳密にアクセス制御済み。

5. 画面構成(完成済み)
① login.html

Firebase Authentication によるログイン

ドメインチェック・名簿チェック完了

成功後→ start.html へ遷移

② start.html(メインメニュー)

途中再開

新規入力(科目登録から)

閲覧モード

ログイン中表示はヘッダー右上へ統一。

③ subjects.html(科目登録画面)
✔ 実装済み機能

学年・コース選択 → 科目一覧読込

担当科目の登録 / 解除

Firestore との同期

「評価基準入力へ」ボタンで evaluation.html へ遷移

ログイン表示の UI 統一済み

✔ 削除機能(最新)

confirm() を廃止し、スタイリッシュモーダルで削除確認を実装

OK → Firestore から該当科目削除

④ evaluation.html(評価基準入力画面)
✔ 完成済みの大機能
🔹 1. 科目選択 → 自動で評価方式・行の初期表示

「考査のみ」選択時 → 期末考査100%を固定で自動生成

「考査+課題」選択時 → 期末考査+空欄1行自動生成

「実験系」選択時 → 空欄1行生成

🔹 2. 保存機能(Firestore)

保存すると evaluationCriteria_2025 / {teacherEmail_subjectId} に保存

保存時に Toast 表示でスタイリッシュな通知

保存後「保存済み」状態で UI がロック

評価方式

行編集

追加行

削除行

🔹 3. 修正機能

「修正する」ボタン押下で再編集可能

修正後は再度保存可能

🔹 4. 科目切替時の徹底管理

保存済/未保存を問わず、
次の科目へ切り替えると UI が完全リセットされるよう改善済み。

🔹 5. 100%チェック

合計が 99〜101% の範囲以外は保存不可

エラーメッセージも UI デザインに準拠し改善済み

6. 実装済み UI 改善一覧

ログイン中の表示位置の統一

評価方式カード UI

合計欄の赤色警告

保存済みボタンのデザイン

削除前確認モーダル(科目登録画面)

科目切替時のロック解除バグ修正

評価方式の自動行生成バグ修正

7. ここまでの導線(確定フロー)
ログイン
   ↓
start.html(途中再開 / 新規入力)
   ↓
subjects.html(科目登録)
   ↓
evaluation.html(評価基準入力)
   ↓(次ステップ)
成績入力画面(これから実装)
   ↓
最終送信

8. 今後のステップ(次チャットで実装する部分)
🎯 次にやるべきは “成績入力画面の設計”
成績入力画面の要件(現時点の確定事項)

評価基準(evaluationCriteria_2025)に基づき入力欄が自動生成される

途中保存ができる(drafts)

保存済かどうかのステータス管理

複数教員が同じ科目を担当する場合の挙動

最終送信後はロック

成績登録済科目を subjects.html で表示する必要があるか検討

データ構造の統一

UI は評価基準入力のデザインラインを踏襲

9. 引き継ぐべき前提情報(重要)

Firestore は年度で分けて管理
→ evaluationCriteria_2025, teacherSubjects_2025, etc.

評価基準は
→ {teacherEmail}_{subjectId}
でキー生成し一意化

科目登録時の subjectId
→ "4_CC_後期_水理学Ⅱb" のような形式

成績入力でも 同じ subjectId を使う

保存済判定は Firestore の該当 doc の存在で可能

💡 10. 次チャットで必要な作業

成績入力画面のために、次チャットでは以下を実施:

✔ 成績入力画面の基本設計
✔ Firestore 構造(成績編)
✔ UI モック
✔ 評価基準との連携処理
✔ 保存 / 修正 / ロック仕様
✨ 完全引き継ぎ書は以上です

この内容があれば、次のチャットで 成績入力画面 を完全に設計できます。

必要があれば、

データ構造図

ER図

画面遷移図

UI ワイヤーフレーム
も作成可能です。