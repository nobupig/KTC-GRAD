📘 完全引き継ぎ書(2025/12/10 最新版・score_input 開発専用)
🎯 目的

このドキュメントは、次のチャットに貼るだけで 何も失われず、正確に続きから開発できるようにするための完全版メモ です。

score_input の分割ファイル化・名簿ロジック・評価基準連携・Step B の位置関係
すべてを「現在の最新版だけ」に整理しています。

🔷 1. 現在のプロジェクト状況(超要約版)

あなたの現在の作業フェーズは:

✔ score_input の外部 JS 分割を開始した段階

score_input_loader.js:完成(あなたが貼り付け済み)

score_input_students.js:これから作成(最重要)

score_input_criteria.js:これから

score_input_modes.js:Step B で実装

score_input_paste.js:Step B-5 で実装

✔ score_input.html には “名簿ロジック + 評価基準 + モード未実装” という状態

つまり、まだロジックが全部1ファイルに残っているので、
これから徐々に JS に取り出す段階です。

🔷 2. Step A(名簿ロジック)完了状況

Step A は 完全に完了 しています。
次の仕様が満たされていることを確認済み。

✔ (1)科目に応じた学生名簿フィルタ

Firestore students のデータ構造:

studentId
name
grade
courseClass(M/E/I/C/A)
number

✔ (2)科目のコース種類に応じた名簿抽出
科目の course	表示すべき学生
M	M の学生のみ
E	E の学生のみ
I	I の学生のみ
C	C と A の学生
CC	C のみ
CA	A のみ
共通(specialType=0)	全学年、コース M → E → I → C → A
✔ (3)並び順

grade(昇順)

coursePriority(M→E→I→C→A)

number(昇順)

✔ 動作確認済み

科目「4CC」「4C」「3A」「1前期共通」などで
正しく名簿が出る状態になっています。

🔷 3. Step B の位置(次にやるべきこと)

Step B は score_input の中核部分で、重要性が高い。

🟩 Step B(次にやること・確定)
B-1:評価基準の割合をヘッダへ反映(最優先)

例:

期末考査 (70%)
課題① (30%)


ヘッダー列を:

学籍番号|学年|組C|番号|氏名|期末考査(70%)|課題①(30%)|最終成績


にする。

B-2:素点モード / 自動換算モード のタブ実装

UI に2つのタブを作成

デフォルトは「自動換算モード」

モード名は以下に統一

素点入力モード

自動換算モード(旧:換算モード)

B-3:入力規則バリデーション
✔ 素点入力モード

各項目の満点 = weight(割合値)
例:70% なら 0~70 の入力のみ可

小数可

マイナス不可

入力拒否・アラート

✔ 自動換算モード

0~100 の値のみ

小数可

weight% が自動適用される(例:70% → 0.7)

B-4:最終成績の計算

weight の合計が 99〜101% → 内部処理では100に正規化

最終成績は floor(切り捨て)

fractional 計算はそのまま保持

B-5:貼り付け入力(studentId でマッピング)

形式例:

91905011,40,30
91905012,45,20


studentId で該当行を探して反映

入力規則に違反 → 全体拒否

🔷 4. 直近のファイル構造(最新版)
public/
  ├── score_input.html   ← まだロジック多い
  ├── js/
  │    ├── score_input_loader.js  ← 完成・貼付済み
  │    ├── score_input_students.js ← 次に作成
  │    ├── score_input_criteria.js ← 必要
  │    ├── score_input_modes.js ← Step B で作成
  │    └── score_input_paste.js ← Step B-5
  ├── evaluation.html
  ├── subjects.html
  ├── start.html
  └── view.html

🔷 5. いま最優先でやるべきこと
✔ score_input_students.js の完全版を作成する(あなたの環境専用)

ここには:

Firestore students の取得

科目の courseClass / specialType / grade によるフィルタ

並び順制御

HTML 行の生成(render)

が入る。

これを作ることで score_input.html を軽量化でき、
以降の Step B〜D が「安全に」移植できるようになる。

🔷 6. 次のチャットへ貼るべき “完全引き継ぎタグ”

以下を次チャットの最初に貼れば、
このドキュメントの内容だけをもとに “正確に続きから開始” します:

【引き継ぎ開始】
score_input 実装ステータス:STEP A 完了 → STEP B 開始前
score_input_students.js 作成から開始したいです。
【引き継ぎ終了】


※ 前の破損原因は “情報過多” です。
この短いタグの方が 100%安全に継続可能 です。