📘 引き継ぎ書
近大高専 成績入力システム(Firebase)
STEP.B 完了時点まとめ(2025年)
1️⃣ プロジェクト概要

Firebase(Firestore)を用いた 教員向け成績入力 Web アプリ

主な機能:

科目選択 → 名簿表示

評価基準に基づく点数入力

自動換算 / 素点モード切替

最終成績・平均点・調整点の表示

途中保存 → 再開(STEP.B)

2️⃣ STEP.B の目的(達成済)
🎯 目的

途中保存 → 再開時に、

すでに入力済みの成績が

最終成績

寄与点表示 ()

平均点・調整点

入力操作なしで正しく表示されること

Firestore reads を増やさない(DOM再構築のみ)

✅ 達成状況

通常科目:完全対応

評価基準が複数ある科目:完全対応

習熟度科目:完全対応(※重要な切り分けあり)

3️⃣ 最終的に確定した仕様(重要)
A. 通常科目(数値評価)

保存 → 再開時:

savedScores を input.value に復元

DOM上で再計算を1回だけ実行

updateFinalScoreForRow

updateAveragePointDisplay

Firestore 再読込なし

B. 習熟度科目(S / A1 / A2 等)

再計算はしない

理由:

updateFinalScoreForRow は input[type="number"] 前提

習熟度は数値評価ではないため、再計算すると表示が消える

保存 → 再開時:

applySavedScoresToTable による 表示の復元のみ

最終成績・表示は保存時の DOM 状態を保持

👉 「習熟度は再計算しない」 が最終確定仕様

4️⃣ 最終的に正しかった実装修正(核心)
score_input_loader.js

handleSubjectChange 内、savedScores 復元部分

// ===== 途中再開:savedScores を input に反映(Firestore reads 追加なし) =====
if (savedScores) {
  // 1) savedScores → input.value へ反映
  applySavedScoresToTable(savedScores, tbody);

  // 2) 通常科目のみ:数値評価の再計算
  if (!isSkillLevel) {
    const rows = tbody.querySelectorAll("tr");
    rows.forEach((tr, index) => {
      updateFinalScoreForRow(tr, criteriaState, modeState, null, index);
    });
  }

  // 3) 平均点・調整点(DOMのみ)
  updateAveragePointDisplay();
}

ポイント

updateFinalScoreForRow は 通常科目のみ

習熟度では 絶対に呼ばない

Firestore reads 増加なし

5️⃣ 今回ハマった最大のポイント(重要な知見)
❌ 誤解されやすかった点

「評価基準が複数あるから壊れる」
→ ❌ 違う

✅ 真因

習熟度科目だけ

再開時に

再計算ロジックを通してしまう

または通していないのに、数値再計算関数を使おうとする

結果:

保存 → 再開で

最終成績が消える

() が消える

しかし

1セル入力

モード切替
→ イベント発火で復活

👉 イベント依存ロジックと、DOM再構築ロジックの混在が原因

6️⃣ 設計として確定した考え方(重要)

途中再開時はイベントに頼らない

表示復元は

DOM再構築を1回だけ

再計算が必要なのは

数値評価のみ

習熟度は

再計算対象にしない

保存時表示を信頼する

これは 教育現場運用としても正しい設計

7️⃣ 現在の到達点
完了済

STEP.A:UI安定化・横スクロール問題解消

STEP.B:途中再開ロジック(完全対応)

次に進めるステップ(候補)

STEP.C:

入力完了判定の明示

未入力行の視覚的強調

保存ボタン活性制御

UX改善:

習熟度と数値評価のUI差別化

再開時メッセージ表示

管理者向け:

入力完了率の可視化

CSV / Excel 出力

8️⃣ 次チャットで最初に書くと良い一文(おすすめ)

「近大高専 成績入力システムです。
STEP.B(途中保存→再開)は 通常科目/習熟度科目ともに完全解決しています。
次は STEP.C 以降の設計・実装を進めたいです。」