📘 **KTC 成績処理システム(Firebase版)

完全引き継ぎ書(2025/12/09 最新版)**

🔵 1. システム構成(確定)

このプロジェクトは Firebase を利用した教務システム。

✔ 利用技術

Firebase Authentication(Googleログイン)

Firestore(NoSQL)

Firebase Hosting

HTML + Vanilla JavaScript

Tailwind風CSS(独自カスタム)

Firestore Security Rules(高セキュリティ)

✔ 認証仕様

@ktc.ac.jp 教員のみログイン可能

Firestore の teachers コレクションの email を基準に判定

学生 / 外部メールは完全拒否

🔵 2. Firestore 構造(確定)
✔ teachers/{email}

教員マスタ

✔ subjects/{subjectId}

科目マスタ

✔ teacherSubjects_2025/{email}

教員×科目の担当割り当て

✔ evaluationCriteria_2025/{subjectId}

科目ごとの今年度の評価基準(1科目1レコード)

項目例

method(評価方式:考査のみ/考査+課題/実験)

items[](評価項目の配列)

totalPercent(合計割合)

isSaved(UI 用フラグ)

✔ evaluationLocks_2025/{subjectId}

編集ロック(排他制御)

✔ evaluationHistory/{id}

評価基準の修正履歴(before/after/理由/編集者など)

✔ drafts/{id}

成績入力の途中保存(後で使用)

🔵 3. 各画面の役割(確定)
✔ index.html

ログイン→自動遷移ハブ

✔ start.html

ホーム画面
(ログイン後のメインメニュー)

✔ subjects.html

担当科目登録

科目読込

追加 / 削除

評価基準入力画面へ遷移

✔ evaluation.html

評価基準入力

評価方式選択

行追加 / 削除

合計100±1%チェック

保存 → Firestore

修正履歴

ロック制御

成績入力画面への遷移ボタン(今回の修正対象)

✔ score_input.html

成績入力画面
(現在 Step1〜2 の UI まで完了。今後本実装)

✔ view.html

教員用の成績閲覧・管理画面(後日)

🔵 4. 直近の致命的バグの修正状況
✔ subjects.html → evaluation.html → score_input.html

遷移ロジックの破損 → 修正済

✔ Copilot 介入により evaluation.html UI が崩壊

→ 復旧済(最新評価基準UIが適用されている)

✔ ログイン状態が消える問題

→ 修正済(遷移ロジックを統一)

✔ 成績入力画面で全学生が表示されてしまう

→ Step1–2 の段階のため、まだフィルタ実装前
(これは仕様どおり)

🔵 5. 現在の 未解決課題(今回発覚した重要ポイント)
❗ 評価基準が保存されているのに、
「途中再開 → 成績入力画面に遷移しない」問題

原因:
evaluation.html の ready 判定が data.isSaved に依存しているため、
保存されていても UI側の条件と合わずボタンが非表示になる。

→ 今回 Stepごとの安全分割指示書にて修正した。

🔵 6. 今回の修正対象(evaluation.html)
🎯 目的

評価基準が正しく登録されている科目の場合のみ、
「成績入力へ」ボタンを表示して score_input.html に飛べるようにする

🔵 7. 完全安全版 Copilot 分割指示書(確定版)

以下をコピペして次チャットへ渡す。

🧩 STEP1(HTMLの確認とミニ修正)
【STEP1:evaluation.html の科目選択エリアに「成績入力へ」ボタンがあることを確認してください】

1. evaluation.html を開いてください。
2. 「担当科目を選択」の <select id="subjectSelect"> を探してください。
3. その直後の <div id="subjectStatus"> の下に、
   以下のボタンが存在することを確認してください。

   <button id="goScoreBtn" class="mini-nav-btn" style="display:none;">成績入力へ</button>

4. 無ければ追加してください。
5. HTMLの他部分(スタイル・ヘッダー等)は変更しないでください。

🧩 STEP2(DOM参照の確認)
【STEP2:goScoreBtn の DOM参照が script 内に存在するか確認】

1. script の先頭付近で次を探してください:

   const goScoreBtn = document.getElementById("goScoreBtn");

2. 無ければ saveBtn の参照直後に追加してください。
3. その他の参照を変更しないこと。

🧩 STEP3(ready 判定ロジックの修正)※最重要
【STEP3:評価基準の「準備完了」判定ロジックを修正】

1. loadCriteria 関数を開いてください。
2. 次のようなブロックを探してください。

   const ready =
     !!data.isSaved &&
     !!data.method &&
     totalPercent >= 99 &&
     totalPercent <= 101;

3. この部分のみ次の内容に置き換えてください。

   const ready =
     !!data.method &&
     totalPercent >= 99 &&
     totalPercent <= 101;

4. if (goScoreBtn) { ... } ブロックはそのまま残し、
   onclick 部分も変更しないでください。

5. その他の処理(updateTotal など)は変更しないでください。

🧩 STEP4(科目変更時の初期化処理は残す)
【STEP4:科目変更時に goScoreBtn を隠す処理を削除しないでください】

1. subjectSelect.addEventListener("change", ...) の中に、
   goScoreBtn.style.display = "none"; が存在します。
2. これは削除・変更しないでください。

🧩 STEP5(保存処理の編集禁止)
【STEP5:保存処理を編集しない】

1. setDoc で isSaved を付けている部分はそのまま残してください。
2. 保存処理(payload)には手を加えないでください。
3. ready 判定だけ STEP3 の変更で実現します。

🔵 8. これで修正後の動作はこうなる
状況	挙動
評価方式が未選択	成績入力ボタン非表示
合計割合が99〜101%	成績入力ボタン表示
未保存(isSaved=falseでもOK)	表示(今回の修正でOKにした)
修正中(total崩壊)	非表示
クリック	score_input.html?subjectId=… へ遷移
🔵 9. 今後の開発ステップ(全体計画)

この後進むべき順番は:

成績入力画面(score_input.html)の本実装

選択科目のフィルタ処理

超過学生のUI実装

調整点/赤点/色付け〜CSV出力

drafts 保存ロジック

Firestoreへの最終成績書き込み

Cloud Functions によるロック自動解除

教務用閲覧画面(view.html)

🔵 10. 引き継ぎ書の使い方

この全文をそのまま次チャットの最初に貼る

その後 Copilot へ **「安全分割指示書 STEP1 から順に適用してください」**と伝える

evaluation.html の修正が完了する

📌 以上が、完全引き継ぎ書(最新版)です

次のチャットでも この引き継ぎ書をそのまま貼るだけ でプロジェクトを確実に継続できます。

必要なら、今回の内容を PDF 化してお渡しすることも可能です。