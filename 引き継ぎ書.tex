📘 **近大高専 成績システム(Firebase 版)

完全引き継ぎ書(最新版 / 2025-12-07)**

① システム全体構成
🔹 1. 使用技術

Front-end:HTML / JS(ES Modules)

Backend:Firebase Authentication / Firestore

Hosting:Firebase Hosting

Storage:今回の範囲では未使用(成績資料等は今後?)

🔹 2. コレクション構造(現時点)
teachers/               ← 教員名簿(name, email など)
subjects/               ← 科目マスタ(年度共通)
teacherSubjects_2025/   ← 教員ごとの担当科目(年度ごとに分割)
settings/period         ← 成績入力期間
notifications/          ← 教務連絡(Slack補完)
evaluation/             ← (今後実装)成績評価基準
evaluationHistory/      ← (今後実装)評価基準修正履歴
drafts/                 ← (今後実装)自動保存用

② subjects.html(科目登録)
🔹 現状できていること
1. ログイン&認証

Firebase Auth(Googleログイン)

教員名は teachers コレクションから取得して表示
 例:「ログイン中:安井宣仁(〜)」

2. 成績入力期間(前期 / 後期 / 期間外)の制御

settings.period の4つの日付を読み取る

期間外は科目登録を非表示にし、閲覧モードだけ提供

3. 科目読み込み

学年・コースから subjects を検索

都市環境(C/CC/CA)は C_ALL として統合表示

前期/後期の条件も cross-check で制御

4. 科目登録(teacherSubjects_年度)

新規登録では 上書きではなく追加

重複 subjectId は自動除去してユニーク化

例:
 1年数学 → 追加
 次に2年物理 → 上書きされず共存する
(この仕様はあなたの希望通り)

5. 教務連絡

科目が一覧にない場合の Slack 補完連絡

notifications へ記録(read: false)

6. 評価基準入力ボタン

科目読み込み横に設置(ヘッダー側のは削除済)

押すと「担当科目一覧を表示するモーダル」が出る

OK を押すと evaluation.html に遷移

登録科目0件なら「まず科目登録してください」

③ evaluation.html(成績評価基準入力)
🔹 現時点で実装済み
1. 科目選択ドロップダウン

subjects.html で登録した担当科目が一覧表示

2. 評価方式の選択 UI(最新仕様)

3種類:

種類	内容
① 考査のみ評価	期末考査100% 固定。項目編集不可。
② 考査+課題	期末考査(固定項目)+課題を追加(%は自由入力)
③ 実験・実習系課題作品	完全自由入力(複数可)

※ UI は 横並び+アイコン+コンパクトなボックス に変更済。

3. 評価項目 UI

行追加(+行追加)

削除

割合の自動サマリー表示(合計: 〜%)

4. 考査のみの場合の制御

期末考査 100% を自動生成

削除ボタンも非表示

行追加ボタンも非表示

5. 考査+課題 の場合

「期末考査」行が自動で1行作成

削除不可

課題追加は自由

6. 実験・実習系の場合

行追加可能

固定項目なし

④ 今後実装すべき内容(あなたの指示ベースで確定)
🔥 A:保存ロジック(必須)

evaluation コレクションに保存形式は以下:

evaluation/{subjectId}
{
  teacherEmail,
  subjectId,
  method,            // exam-only | exam-plus | practical
  items: [
    { label: "...", percent: 30 },
    ...
  ],
  total,             // 自動計算
  updatedAt,
  updatedBy
}

🔥 B:評価基準の入力ルール(確定)
1. 合計値チェック

99〜101 まで許容(±1%誤差)

それ以外は保存不可

2. 複数担当科目の場合

evaluation に既に存在する場合 → 編集ロック

画面上部に警告:

※ この科目は複数教員が登録しています。
  評価基準は既に ○○先生 が入力済みです。


(将来:修正依頼ボタンの追加も可能)

3. 特別科目のスキップ制御

subjects.html で:

科目が special === true の場合
→ evaluation.html をスキップして成績入力画面へ

4. 途中保存

将来:

drafts/{teacherEmail}/subjectId

⑤ UI に関する修正の方向性(評価方式のカード改善済)
評価方式をもっとコンパクトに

アイコン+短い名称

折り返し防止

ラジオとアイコンの距離調整済

コンテナの height/border-radius/padding を調整済

⑥ 現在の課題(あなたからの依頼)

ログイン中:〜 の位置が evaluation.html で崩れている
 → Copilot 指示書で再調整中

評価方式選択カードのさらなる微調整
 → フォント・余白の最適化

evaluation.html ← subjects.html 「戻る」ボタンの位置整理
 → こちらも微調整済だが最終整形が必要

保存機能はまだ未実装
 → 次チャットで実装予定

⑦ 次のステップ(保存関連整備)

次チャットで以下を実装:

🔥 ステップ①:評価基準の保存実装(evaluation コレクションへの write)

選択中科目の subjectId ごとに保存

新規保存/上書き保存の区別

±1%誤差チェック

入力方式ごとのバリデーション分岐

🔥 ステップ②:複数教員が1科目を扱う場合のロック

evaluation/{subjectId} が存在した場合
 → isLocked = true
 → 編集不可+警告表示

🔥 ステップ③:subjects.html → 特別科目の制御

詳細仕様に基づき、特別科目なら evaluation.html をスキップして
 → grades.html(成績入力画面)へ遷移

📌 次チャットで必要なこと

あなたは「保存機能」をどうしたいかだけ選んでください。

A方式:科目ごとに完全上書き
B方式:draft(途中保存)+ publish(確定保存)
C方式:年度別(2025→2026)、評価基準のコピーも継承

どれで実装したいですか?

必要であれば
subjects.html / evaluation.html の最新版を再添付しての指示書生成 も可能です。

次チャットで保存仕様に進むため「保存方式の選択」だけお願いします。