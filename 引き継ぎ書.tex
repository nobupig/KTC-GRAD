📘【完全引き継ぎ書(最新版)—全文】

(※ まずは全文を提示します。後でファイル化します)

🔷 1. プロジェクト概要(score_input.html 中心)

本プロジェクトは以下を主目的としています:

Firebase+Firestore 基盤による成績処理 Webアプリ

旧 GAS 版の再実装ではなく、Firebase 版として再構築

主に教員が扱う画面(subjects, evaluation, score_input)を段階実装

とくに score_input.html は最も複雑な画面であり、次の機能搭載を目的とする

🔷 2. score_input.html の最終目標機能一覧
❶ 成績評価基準の反映

evaluationCriteria_yyyy/{subjectId} を読み取り

項目名+割合(%) をヘッダに出す(期末考査(70%) など)

❷ 科目に応じた名簿フィルタ

年度、学年、コース、科目名から適切な学生一覧を表示

CA → C と A 学生両方

CC → C のみ

CA → A のみ

共通科目(specialType=0)は「M → E → I → C → A」の順で並ぶ

その他コース科目は grade と courseClass で絞って並び替える

最終並び順は grade → coursePriority → number

❸ 評価基準未登録科目が存在すると成績入力不可(強制モーダル)

1 科目でも評価基準未登録 → score_input の UI をロックし

「評価基準を入力してください」モーダルを表示

evaluation.html に強制遷移 OR モーダル閉じ不可

❹ 成績入力モード

素点モード

換算モード(自動換算モードという名称に変更予定)

❺ 入力規則・バリデーション(モードごと)

素点モード:

各項目の満点は評価基準の「割合値」
例:期末考査70% → 入力可能値は 0~70

小数点は可

マイナス不可

換算モード:

0~100 の値を入力(小数点可)

内部で weight% を乗算して合計を自動計算

weight を 100 に正規化 (±1% 許容)

❻ 最終成績は常に小数点切り捨て
❼ Excel(xlsx)出力(旧版踏襲)
❽ 超過学生管理(後半 STEP)

モーダルで overHours 入力

over(超過) → 色付け

failing(赤点) → 色付け

両方 → 別色

🔷 3. Step A〜D の進捗整理
✔ STEP A:完了(名簿表示まで)

進捗:

evaluation 判定

科目情報読み込み

名簿表示(grade, courseClass → 科目ルールに応じたフィルタ)

並び:grade → coursePriority → number の適切表示

CC / CA / C / A の特別ルール実装

共通科目順(M→E→I→C→A)実装済

→ 名簿表示は完全完了

✔ STEP B:いまから実装すべき領域(最重要)
STEP B-1

評価基準の割合を score_input ヘッダに反映
例:
期末考査 (70%)
課題① (30%)

STEP B-2

素点モード/換算モード(正式名称:自動換算モード)切替タブ

STEP B-3

入力規則バリデーション

素点モード:0〜weight

自動換算:0〜100

貼り付けでも同様に検証し、違反時は alert → 反映拒否

STEP B-4

最終成績計算

normalize(99〜101% → 100)

計算後、小数点切り捨て

STEP B-5

貼り付け入力(複数行対応)
例:

9220001, 50, 30  
9220002, 40, 50  


→ studentId で対応行を検索して反映

✔ STEP C:後で(超過学生・赤点判定)
✔ STEP D:最後(Excel 出力 + UI 仕上げ)
🔷 4. 名簿ロジック再整理(最終版)
科目 specialType(仮)

specialType=0 → 共通 → 全学年共通 + M, E, I, C, A 全員
specialType≠0 → 専門科目

専門科目の course ルール

C → C + A

CC → C のみ

CA → A のみ

M/E/I → 当該コースのみ

並び順

grade(昇順)

coursePriority(M→E→I→C→A)

number(昇順)