📘【最新版】成績入力画面(score_input.html)
STEP別 Copilot 完全指示書(保存版・コピペ可)
============================
🟦 STEP1:学生一覧+評価項目の最小表示
============================
✅ STEP1-1:学生の基本5列を表示する
🎯 目的

score_input.html に「学籍番号、学年、組コース、番号、氏名」を一覧表示する。

📝 Copilot 指示(コピペ用)
これは score_input.html の STEP1-1 の作業です。

【目的】
Firebase の students コレクションから学生データを取得し、
以下の 5 列をテーブルに表示してください:
- 学籍番号
- 学年
- 組・コース
- 番号
- 氏名

【禁止事項】
・score_input.html の既存構造を壊さない
・計算ロジックを追加しない
・選択科目・習熟度・超過・赤点など一切実装しない
・evaluation.html には絶対に触れない

【必須】
・差分実装のみ
・renderStudents() のような関数を新規追加する形で OK

以上の条件で安全なコードのみ提案してください。

✅ STEP1-2:評価項目(項目名だけ)を表示する
🎯 目的

evaluationCriteria_2025/{subjectId} の項目名を列ヘッダに表示。
計算はまだ不要。

📝 Copilot 指示
これは score_input.html の STEP1-2 の作業です。

【目的】
evaluationCriteria_2025/{subjectId} から評価項目名を取得して、
テーブルの列ヘッダに反映してください。
学生行には項目数に応じた input を作るだけでOKです。
計算ロジックはまだ追加しません。

【禁止事項】
・割合計算、合計列の追加は禁止
・既存コードの削除・上書きを禁止
・evaluation.html を参照・修正しない

【必須】
・差分のみ反映
・loadCriteria() のような小さな関数を新規追加してください

安全な差分コードのみ提案してください。

============================
🟦 STEP2:素点モードの合計計算(割合の正規化含む)
============================
✅ STEP2-1:割合の ±1% 正規化を実装
🎯 目的

評価基準合計が 99〜101% の場合 → 100% に正規化して計算で使用。

📝 Copilot 指示
これは score_input.html の STEP2-1 の作業です。

【目的】
evaluationCriteria_2025 の割合を取得し、
合計が 99〜101 の場合は内部で 100% に正規化する
normalizeWeights() 関数を追加してください。

【禁止事項】
・UI 変更を行わない
・モード切替や赤点判定はまだ実装しない
・既存関数を上書きしない

【必須】
・normalizeWeights(weights[]) を新規追加
・結果の normalizedWeights を返すだけの関数にする

差分のみ提案してください。

✅ STEP2-2:素点モードの合計計算を実装
🎯 目的

「素点 × 正規化割合」で合計を算出する。

📝 Copilot 指示
これは score_input.html の STEP2-2 の作業です。

【目的】
素点モードの合計計算を実装します。
合計式:SUM( inputValue * normalizedWeight / 100 )

【禁止事項】
・換算モードはまだ実装しない
・赤点判定や超過処理を追加しない
・既存コードの書き換え禁止

【必須】
・calculateTotalForRow() のような関数を新規追加
・input の onInput でリアルタイム計算を行う

差分コードのみ提案してください。

============================
🟦 STEP3:モード切替+注意書き
============================
✅ STEP3-1:素点/換算モードの切替タブを作成
🎯 目的

UI 上部にモード切替タブを追加する。

📝 Copilot 指示
これは score_input.html の STEP3-1 の作業です。

【目的】
画面上部に「素点モード」「換算モード」の切替タブを追加し、
クリックで currentMode を変更できるようにしてください。

【禁止事項】
・計算ロジックの変更は禁止(まだ換算モード未実装)
・既存のUI構造を壊さない
・evaluation.html を触らない

【必須】
・currentMode = "raw" | "scaled" のように状態管理
・タブのCSSは簡易追加のみ

差分コードを提案してください。

✅ STEP3-2:換算モードの計算ロジックを実装
🎯 目的

素点を割合に応じて評価(満点制は後で拡張可)。

📝 Copilot 指示
これは score_input.html の STEP3-2 の作業です。

【目的】
換算モードを実装します。
素点を normalizedWeight で再評価し、点数を算出してください。
処理は素点モードと同等で構いません。

【禁止事項】
・素点モードを壊さないこと
・他の STEP の機能を混ぜないこと
・既存関数の破壊禁止

【必須】
・currentMode を条件分岐に利用

差分のみ提案してください。

✅ STEP3-3:注意書き(折りたたみUI)
🎯 目的

旧版の注意書きを「折りたたみ表示」で追加。

📝 Copilot 指示
これは STEP3-3 の作業です。

【目的】
素点/換算モードの説明と入力注意事項を、
折りたたみ式パネルとして score_input.html に追加してください。

【禁止事項】
・UIレイアウトを大幅に変更しない
・説明内容以外のコードを編集しない

【必須】
・デフォルトは閉じる(アコーディオンUI)
・CSS は最小限追加のみ

安全な差分コードを提案してください。

============================
🟦 STEP4:平均点・調整点(fMark)・赤点判定
============================
✅ STEP4-1:平均点計算
📝 Copilot 指示
これは STEP4-1 の作業です。

【目的】
各学生の合計点から平均点を算出し、
画面上部に「平均点 = xx.x」を表示してください。

【禁止事項】
・調整点(平均×0.7)はまだ実装しない
・赤点判定はまだ行わない

差分コードのみ提案してください。

✅ STEP4-2:調整点(平均×0.7)+ fMark による赤点基準作成
📝 Copilot 指示
これは STEP4-2 の作業です。

【目的】
subjects.{subjectId}.fMark を読み取り、
調整点の有無で赤点ライン failingThreshold を計算してください。

【仕様】
・fMark = "○" → 調整点なし(赤点=60)
・それ以外 → 調整点 = ceil(平均×0.7)

【禁止事項】
・既存表示の破壊禁止
・evaluation.html には触れない

差分コードのみ提案してください。

✅ STEP4-3:赤点学生の行色付け
📝 Copilot 指示
これは STEP4-3 の作業です。

【目的】
合計点 < failingThreshold の学生行に
CSS クラス .failing を付与して背景色を変えてください。

【禁止事項】
・超過と混ぜたロジックはまだ実装しない
・CSS は最小限追加のみ

差分コードのみ提案してください。

============================
🟦 STEP5:超過学生(モーダル+色分け)
============================
✅ STEP5-1:超過学生登録モーダル
📝 Copilot 指示
これは STEP5-1 の作業です。

【目的】
超過学生を登録するモーダルUIを追加します。
各学生について overHours を入力して保存します。

【禁止事項】
・行の色付けはまだ行わない
・score_input.html 以外のファイルを触らない
・Firestore 本保存はまだ禁止(draft のみ)

差分コードを提案してください。

✅ STEP5-2:超過学生フラグ保存(draft)
📝 Copilot 指示
これは STEP5-2 の作業です。

【目的】
モーダルで入力された超過学生の情報を
scoreDrafts_2025 に保存できるようにしてください。

【禁止事項】
・最終保存(scores_2025) はまだ行わない
・evaluation.html を触らない

差分コードのみ提案してください。

✅ STEP5-3:超過/赤点/両方の行色分け
📝 Copilot 指示
これは STEP5-3 の作業です。

【目的】
行に以下のクラスを付けて色分けしてください:
・超過 → .over-student
・赤点 → .failing
・両方 → .over-and-failing

【禁止事項】
・既存の行レンダリングを破壊しない
・CSS は minimal に追加するのみ

差分コードを提案してください。

============================
🟦 STEP6:選択科目(履修者のみ表示)
============================
📝 Copilot 指示
これは STEP6 の作業です。

【目的】
subjects.{subjectId}.isElective が true の場合は、
enrollment データから履修登録者のみ学生一覧に表示してください。

【禁止事項】
・学生全体取得ロジックを破壊しない
・計算処理は触らない

差分コードのみ提案してください。

============================
🟦 STEP7:習熟度(列追加+モーダル+ソート)
============================
✅ STEP7-1:習熟度列の追加
📝 Copilot 指示
これは STEP7-1 の作業です。

【目的】
isProficiencyCourse=true の科目に限り、
学生情報列に「習熟度」列を追加してください。

【禁止事項】
・表示順を壊さない
・習熟度の編集ロジックはまだ不要

差分コードのみ提案してください。

✅ STEP7-2:習熟度入力モーダル
📝 Copilot 指示
これは STEP7-2 の作業です。

【目的】
習熟度(S/A1/A2/A3)を一括登録できるモーダルを追加します。
直接入力とコピペ貼り付け両方に対応してください。

【禁止事項】
・ranking・sort の実装はまだ
・Firestore 本保存は禁止(draft のみ)

差分コードのみ提案してください。

✅ STEP7-3:習熟度でソート
📝 Copilot 指示
これは STEP7-3 の作業です。

【目的】
習熟度列をクリックした際に、
S → A1 → A2 → A3 → その他 の順でソートしてください。

【禁止事項】
・既存行構造を壊さない
・sort ロジックは専用関数として分離する

差分コードを提案してください。

============================
🟦 STEP8:素点一括貼り付け
============================
📝 Copilot 指示
これは STEP8 の作業です。

【目的】
Excel/スプレッドシート形式の
「学籍番号,点数...,点数」形式の貼り付けをパースし、
該当行の input にまとめて反映する機能を追加します。

【禁止事項】
・既存の input 構造を壊さない
・計算ロジックは既存関数を呼び出すだけ
・UIを大きく変更しない

差分コードのみ提案してください。

============================
🟦 STEP9:成績Excel出力(旧版踏襲)
============================
📝 Copilot 指示
これは STEP9 の作業です。

【目的】
成績一覧を Excel(xlsx) 形式で出力する機能を追加します。
旧システムのフォーマットを踏襲し、
以下の構成で出力してください:

1行目:科目名(2列結合)
2行目:平均点
3行目:調整点
4行目:成績/最終成績
5行目以降:学生データ

【禁止事項】
・既存テーブル構造を壊さない
・scores_2025 の保存ロジックを変更しない
・超過や赤点の判定ロジックを書き換えない

差分コードのみ提案してください。

============================
🟦 STEP10:UI調整・仕上げ
============================
📝 Copilot 指示
これは STEP10 の仕上げ作業です。

【目的】
UIの微調整、説明文の整備、不要ログ削除、CSSの極小改善を行います。

【禁止事項】
・既存ロジックを壊す変更は絶対禁止
・ファイル全体の置換禁止

差分コードのみ提案してください。

📌 最終確認
✔ すべてのSTEPは score_input.html 前提
✔ evaluation.html は一切触らない
✔ 各STEPを “そのまま Copilot に貼るだけ” で安全に実装可能
✔ 誤爆を防ぐ「禁止事項」が明記されている