📘【STEP B — score_input 機能実装 詳細仕様書(2025/12/10最新版)】

score_input の中核機能を実装するための 仕様書の確定版 です。
このまま Copilot に投げても良いですし、あなたが読んでも分かりやすいように整理しています。

🟦 STEP B — 成績入力画面の中核実装(最重要フェーズ)
🔷 B-0:共通ルール(重要)

この STEP B は score_input の以下ファイルに分割して作成:

js/
 ├ score_input_students.js     ← STEP A 完了済(名簿)
 ├ score_input_criteria.js     ← ★STEP B-1 で実装
 ├ score_input_modes.js        ← ★STEP B-2〜B-4
 └ score_input_paste.js        ← ★STEP B-5


score_input.html は ロジックを書かず、UI(HTMLの器)だけを保持。

すべての JS は 差し込み式(追加で破壊しない)で実装。

🟩 B-1:評価基準の割合をヘッダーに反映(最優先)
🎯目的

evaluationCriteria_2025/{subjectId} を読み取り、
項目名 + (割合%) を score_input のテーブルヘッダに表示する。

▼ 表示例

評価基準:

期末考査:70%
課題①:30%


成績入力テーブル:

学籍番号 | 学年 | 組・コース | 番号 | 氏名 | 期末考査(70%) |課題①(30%)| 最終成績

▼ 実装内容

score_input_criteria.js に以下を実装:

Firestore から evaluationCriteria を取得

正規化(±1%)はこの段階ではまだしない

DOM のヘッダ行(thead)に動的挿入

▼ UI 仕様

最終成績列は固定

評価基準の項目数に応じて動的列追加

項目名は最大12文字程度まで改行無し

🟩 B-2:モード切替タブ(素点/自動換算)
🎯目的

教員が迷わないよう、2つのモードで入力方法を選択可能にする。

▼ モード名(正式名称)

素点入力モード(raw mode)

自動換算モード(scaled mode)

▼ UI仕様

score_input.html の上部にタブを設置

視覚的には「オン/オフ切り替えボタン」形式

デフォルトは 自動換算モード(scaled)

▼ 動作

タブクリック → currentMode = 'raw' | 'scaled'

モード切り替え時に入力欄の placeholder も切り替え

例:
raw → “0〜70 の範囲で入力”
scaled → “0〜100 の範囲で入力”

🟩 B-3:入力バリデーション(モードごと)
🎯目的

誤操作や破壊入力を完全に防止する。

✔【素点入力モード】

各項目の満点 = weight(割合値そのまま)

例:70% → 0〜70 の範囲で入力

▼ 必須ルール

小数点可(例:15.5)

マイナス不可

weight(割合)を超える数値は不可

入力違反 → アラート+反映キャンセル

✔【自動換算モード】

入力可能範囲 = 0〜100
入力された値に weight を乗算して内部素点を生成

例:60(入力) × 70% → 42点

▼ 必須ルール

小数点可

100超え不可

マイナス不可

✔【共通バリデーション仕様】

入力違反 → アラート(スタイリッシュ版に変更予定)

その場で元の値に戻す

貼付け時にも同じバリデーションを適用

🟩 B-4:最終成績計算(切り捨て+±1%正規化)
🎯目的

どの科目でも統一した評価処理ができるようにする。

✔ 1. 重み(weight)の正規化(normalize)

評価基準の割合合計が:

合計	動作
99〜101%	100 に正規化して扱う
上記以外	評価基準の誤りとして警告(別途モーダル)
✔ 2. 最終成績の計算式
final = floor( SUM( itemValue * normalizedWeight / 100 ) )


raw モード → 自分で割合換算して入力

scaled モード → 100入力 → 自動換算

最終成績は必ず 小数点切り捨て。

🟩 B-5:貼り付け入力(複数行対応)
🎯目的

教員が Excel からコピペした値を一括反映できるようにする。

✔ 入力形式(想定)
91905011,50,30
91905012,45,35.5
91905013,60,40


studentId と項目数が一致しない場合 → 全行拒否

少数、小数点OK

バリデーション適用(weight超過など)

✔ ステップ

score_input_paste.js に以下を作成:

paste イベントでテキスト取得

行ごとに split

studentId で該当行を検索

input 要素へ set

モードごとのバリデーション適用

最終成績再計算

🟦 STEP B の最終ゴール(完成状態イメージ)
[ 素点入力 ] [ 自動換算 ]  ← タブ

科目名:材料力学Ⅰ(1G)   | 評価基準:期末(70%)・課題(30%)
平均:62.1   調整点:44   赤点基準:60(表示のみ)

┌──────────────────────────────────────┐
│ 学籍番号|学年|組|番号|氏名|期末考査(70%)|課題(30%)|最終成績 │
│ 91905011|5|I|1|青海| 48.5| 20  | 68 │
│ 91905012|5|I|2|赤坂| 50  | 23  | 73 │
└──────────────────────────────────────┘

🔷 この仕様書を次チャットへ引き継ぐ方法

以下の 3 行を次のチャットに貼るだけで OK:

【引き継ぎ開始】
score_input 実装ステータス:STEP A 完了 → STEP B 開始
STEP B(詳細仕様書)に従って実装を進めたいです。
【引き継ぎ終了】
