📘【Step B 詳細仕様書(全文)】
Step B の目的

score_input.html に 成績計算 UI の基盤 を完成させる。

🔷 STEP B-1:ヘッダに評価基準(weight)を表示
やること

evaluationCriteria_{year}/{subjectId} を読み込む

each item: { label, percent }

ヘッダーに追加:

例:

<th>期末考査 (70%)</th>
<th>課題① (30%)</th>
<th>最終成績</th>

必要な差分

HTML:ヘッダー生成部分を動的にする

JS:renderHeader() を作って反映

🔷 STEP B-2:素点モード/自動換算モード 切替タブ
UI

タブ2種

「素点モード」

「自動換算モード(旧:換算モード)」

動作

currentMode = "raw" / "scaled"

切替時は全 input の再計算を実行

🔷 STEP B-3:入力規則バリデーション
素点モード

0〜weight

小数点 OK

マイナス NG

貼り付け時も同じ規則で検証し、エラー時は alert

自動換算モード

0〜100

小数点 OK

マイナス NG

🔷 STEP B-4:最終成績計算
流れ

weights[] を読み込み

合計 99〜101 → normalize to 100

各項目の計算

合計値は Math.floor で切り捨て

最終成績セルに反映

素点モード
itemScore * (weight / weightSum)

自動換算モード
(rawInput / 100) * weight

🔷 STEP B-5:貼り付け入力機能(大量入力対応)
形式
学籍番号, score1, score2, ...
学籍番号, score1, score2, ...

処理

貼り付けされた block を改行 split

studentId で該当行を検索して反映

バリデーション NG の場合 → alert → 全反映中止