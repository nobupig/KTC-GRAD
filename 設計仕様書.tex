🔵 【Firebase成績管理システム 統合仕様書(第C稿)】
=====================================
0. システム概要
=====================================

本アプリは、教員が担当科目を登録し、
成績評価基準の設定・成績入力を行い、教務へ提出するためのシステムである。

データ管理は Firebase(Firestore)、
認証は Firebase Authentication を使用する。

旧システム(GAS)で運用されていた仕様を忠実に継承しつつ、
Firebase向けに安全性・拡張性・操作性を最適化したもの。

=====================================
1. ログイン/認証仕様
=====================================
1-1. 認証

Firebase Authentication による Googleログイン

ログイン成功後、email を取得

1-2. ログイン制限(教員のみ)

以下を満たした場合のみログイン許可:

✔ 必須条件

メールアドレスが @ktc.ac.jp

Firestore の teachers コレクションに email が登録されている

❌ 拒否対象

g+数字@ktc.ac.jp(学生)

s+数字@ktc.ac.jp(学生)

個人メール

名簿にない教職員アドレス

エラーメッセージ
このメールアドレスではログインできません。
@ktc.ac.jp の教員アカウントでログインしてください。

1-3. Firestore セキュリティルール

学生メールは read/write 全て拒否

教員メール+teachersコレクション登録済みメールのみ権限付与

=====================================
2. 成績入力期間(前期/後期)
=====================================
2-1. 教務が設定する項目

Firestore の config に以下のように登録:

gradeInputPeriod_2025_first { start, end }
gradeInputPeriod_2025_second { start, end }


年度ごとに前期/後期の期間を変更できる。

2-2. システム制御

期間内 → 成績入力・評価基準編集が可能

期間外 → 編集不可、修正依頼のみ
(修正依頼は後述の evaluationPermissions で制御)

=====================================
3. 科目一覧(静的情報)
=====================================

教務側が準備する「科目一覧」シートを Firestore の subjects に保持。

subjects/{subjectId}
{
  name, grade, course, term(前期/後期/通年), etc...
}


担当教員情報は 保持しない(現行運用仕様)。

=====================================
4. 科目登録(教員が前期/後期で毎回登録)
=====================================
4-1. 年度+期ごとの登録

例:

2025前期 → 教員が担当科目を登録

2025後期 → 再度登録
→ 前後期は完全に別扱い

4-2. 科目一覧の表示条件
期間	表示される科目
前期	前期科目のみ
後期	後期 + 通年科目
4-3. 科目登録画面 UI

学年(プルダウン)

コース(プルダウン)

条件に一致した科目一覧をテーブル表示

教員が ✔ を入れて複数科目を一括選択

「科目を登録する」ボタン押下

🔵 4-4. 最終確認モーダル(新仕様)

「科目を登録する」クリック後、必ず下記モーダルを表示:

モーダル内容
以下の科目を登録します:

・数学I(1年C)
・英語A(1年C)
・物理Ⅱ(2年A)

この内容で間違いありませんか?

ボタン

OK(確定して登録)

戻る(修正する)

=====================================
5. 科目登録後に Firestore に保存される形式
=====================================
5-1. termSubjects(年度+期+科目)
termSubjects/{year_term_subjectId}
{
  year,
  term,
  subjectId,
  isMultiTeacher,
  createdAt,
  updatedAt
}

5-2. 担当教員
termSubjects/{id}/teachers/{teacherEmail}
{
  email, name, registeredAt
}

5-3. 複数担当判定(自動)

教員数 >= 2
→ isMultiTeacher = true

これにより 誤登録も検知可能。

=====================================
6. 科目一覧に無い科目の教務連絡
=====================================

科目登録画面に「科目が一覧にない場合はこちら」ボタンを配置。

入力項目

学年

コース

科目名(自由記述)

備考

送信先

メール → nyasui@ktc.ac.jp

必要なら Firestore にログとして保存

missingSubjects/{autoId}
{
  grade, course, subjectName, note, teacherEmail, createdAt
}

=====================================
7. 成績評価基準(evaluation)
=====================================
7-1. 入力条件

科目登録済みであること
(未登録の科目は評価基準入力不可)

7-2. UI(科目タブ方式)

複数科目を持つ教員は、画面上部タブを切り替えて入力。

7-3. 評価区分

2分類:

A. 考査実施科目

考査のみ → 自動100%

考査+課題 → 合計100%必須(±1% 誤差許容)

B. 実験・実習・課題作品

複数課題入力可

合計100%必須(±1% 誤差許容)

=====================================
8. 複数担当科目(誤登録検知)
=====================================
8-1. isMultiTeacher = true の場合

評価基準画面上部に警告:

※ この科目は複数の教員が登録しています。
 誤登録の可能性がある場合、教務へご連絡ください。


教務連絡ボタンの導入は任意。

=====================================
9. 成績評価基準の修正(あなたが選んだ仕様)
=====================================
9-1. 修正ボタンは教員全員に表示
9-2. 修正理由入力が必須

理由を入力しないと編集モードに遷移不可。

9-3. 修正確定時
updatedBy
updatedAt
lastReason


を更新し、履歴を保存:

evaluationHistory/{id}
{
  before, after,
  updatedBy, updatedAt,
  reason
}

9-4. 入力期間外の扱い

評価基準の直接編集は禁止
→ 「修正依頼」ボタンを表示

=====================================
10. 途中保存(draft)
=====================================

科目登録後
評価基準入力画面
成績入力画面(後で仕様策定)

で draft 保存が可能。

termSubjects/{id}/drafts/{teacherEmail}

=====================================
🔵【現在の仕様まとめ(最終版 第C稿)】
=====================================

この設計書は、
これまであなたの説明・要件を完全に整理し反映した最新版です。

ログイン制御(教員だけ)

前期/後期の成績入力期間管理

科目登録

最終確認モーダル(今回追加)

科目一覧に無い科目の教務連絡

年度+期ごとの担当管理

複数担当の自動判定

成績評価基準入力(考査/実習)

修正理由必須

履歴管理

複数担当時の警告表示

途中保存

管理者による修正依頼対応

現時点で決まっている全仕様を網羅しています。